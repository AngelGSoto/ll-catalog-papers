\pdfoutput=1
\RequirePackage{amsmath}
\documentclass[iop, apj]{emulateapj}
\usepackage[varg]{newtxmath}
\usepackage{newtxtext}
\usepackage[spanish,es-minimal,english]{babel}
\usepackage[utf8]{inputenc}
\usepackage{natbib}
\usepackage{microtype}
\usepackage{hyperref}
\graphicspath{ {figs/}, {../}, {../luis-programas}}
\bibliographystyle{apj}


%% Commands for the postage stamp images
\setlength{\fboxsep}{0pt}%
\newlength\figwidth
\setlength\figwidth{0.32\textwidth}
\newlength\figstampcolsep
\setlength\figstampcolsep{5pt}
\newcommand\BowshockFig[1]{
  \includegraphics[width=\figwidth, clip, trim=60 50 100 50]
  {#1}
}
\newcommand\raiselabel[1]{\raisebox{0.5\figwidth}[-0.5\figwidth]{#1}}

\newcommand\oiii{[\ion{O}{3}]}
\newcommand\nii{[\ion{N}{2}]}
\newcommand\sii{[\ion{S}{2}]}
\newcommand\heii{[\ion{He}{2}]}
\newcommand\ha{\ensuremath{\mathrm{H\alpha}}}
\newcommand\hb{\ensuremath{\mathrm{H\beta}}}
\newcommand\hg{\ensuremath{\mathrm{H\gamma}}}
\newcommand\elec{\ensuremath{_{\mathrm{e}}}}
\newcommand\Te{\ensuremath{T\elec}}
\newcommand\Ne{\ensuremath{n\elec}}
\newcommand\Wav[1]{\ensuremath{\lambda #1}}
\newcommand\thC{\ensuremath{\theta^1\,\mathrm{Ori~C}}}

\renewcommand\clearpage{}

\begin{document}

\subsection{Interproplyd shells}
\label{sec:interproplyd-group}

The interproplyd shell is a  group of small proplyds, which they are associated with small emission arcs which are formed by the interactions between two individual photoevaporation flows.

\textit{066-652.} This object was previously classified as irregular by \citet{ODell:1996a}. It was reported as a pre-pain-Sequence binaries by \citet{Reipurth:2007} and later cataloged as a proplyd and binary system by \citet{Ricci:2008a}. An small emission arc is associated with this proplyd, which is produced by the interaction with 066-652N flow. 
    
\textit{160-350.} This was previously reported as an irregular proplyd by \citet{Odell:1994} and was designated 159-350. Later, \citet{Ricci:2008a} cataloged it as a binary system and ionized disk seen in emission. The 159-350 outflow interact with the 160-350 flow producing a small emission arc.  

\textit{162-456.} \citet{ODell:1996a} cataloged this object as a stellar source and designated it 162-457. It was listed as a pre-main-sequence binary star by \citet{Reipurth:2007}. A small emission arc is formed by the interaction of the flows come from 162-456 and 162-456NE, the latter located in the Northeast of 162-456.     

\textit{168-326N.} This source was previously reported as an unclear object due to saturation by \citet{Odell:1994}. Later, it was catalogued as a proplyd and binary system by \citet{Ricci:2008a}. Comparing the 5 GHz radio emission to the \ha{} and \oiii{} $\lambda$5007 emission is perceptive a radio source between southeast radio source 168-326SE and the northwest radio source 168-326NW. This structure is probably an interacting zone between the two components \citep{Graham:2002}. From Fig. ?? in not possible to see any emission arc because the system is very affected by saturation.    

\textit{173-342.} This was classified as a semi-stellar and elongated object with diffuse boundary and appears designated 173-341 \citet{Odell:1994}. This source was identified as a proplyd and binary system by \citet{Ricci:2008a}. The emission arc is probably formed by the interaction of 173-342 with 177-341 (HST1). The arc emission shock in clearly visible in the image. 

\textit{175-321.} This was catalogued as a stellar source by \citet{ODell:1996a}. The emission arc is not produced by the interaction of two proplyd flows. However, the shock seems to be oriented towards th1D, which is very close to it. This arc has not been reported in the literature.

\textit{204-330.} This source was previously reported as a proplyd and binary system by \citet{Ricci:2008a} on which appears with the designation 204-330.  Due to the interaction between the flows of 204-330 and 205-330 proplyds is formed small and faint emission arc. This shock has also not been reported in the literature.

\clearpage
\subsection{New arc discoveries}
\label{sec:problematic-group}


\label{sec:notshell}

\textit{4417-653.} 

\textit{4491-627.} 

\textit{4550-659.} 

\textit{4531-628.} 

\textit{4520-419.} 

\textit{4505-336.} 


\end{document}