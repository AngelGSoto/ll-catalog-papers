\pdfoutput=1
\RequirePackage{amsmath}
\documentclass[iop, apj]{emulateapj}
\usepackage[varg]{newtxmath}
\usepackage{newtxtext}
\usepackage[spanish,es-minimal,english]{babel}
\usepackage[utf8]{inputenc}
\usepackage{natbib}
\usepackage{microtype}
\usepackage{hyperref}
\graphicspath{ {figs/}, {../}, {../luis-programas}}
\bibliographystyle{apj}


%% Commands for the postage stamp images
\setlength{\fboxsep}{0pt}%
\newlength\figwidth
\setlength\figwidth{0.32\textwidth}
\newlength\figstampcolsep
\setlength\figstampcolsep{5pt}
\newcommand\BowshockFig[1]{
  \includegraphics[width=\figwidth, clip, trim=60 50 100 50]
  {#1}
}
\newcommand\raiselabel[1]{\raisebox{0.5\figwidth}[-0.5\figwidth]{#1}}

\newcommand\oiii{[\ion{O}{3}]}
\newcommand\nii{[\ion{N}{2}]}
\newcommand\sii{[\ion{S}{2}]}
\newcommand\heii{[\ion{He}{2}]}
\newcommand\ha{\ensuremath{\mathrm{H\alpha}}}
\newcommand\hb{\ensuremath{\mathrm{H\beta}}}
\newcommand\hg{\ensuremath{\mathrm{H\gamma}}}
\newcommand\elec{\ensuremath{_{\mathrm{e}}}}
\newcommand\Te{\ensuremath{T\elec}}
\newcommand\Ne{\ensuremath{n\elec}}
\newcommand\Wav[1]{\ensuremath{\lambda #1}}
\newcommand\thC{\ensuremath{\theta^1\,\mathrm{Ori~C}}}

\renewcommand\clearpage{}

\begin{document}

\section{Newest bow shock description}
\label{sec:description}

\subsection{Southeast group}
\label{sec:se-group}

\textit{169-338.} This is a small and faint previously reported proplyd \citep{ODell:1994a, Ricci:2008a} with a well defined tail. The emission arc associated to this proplyd is very faint and clumpy but well-defined. 

\textit{189-329.} The central source was first classified as star by \citet{ODell:1996a} and later as a proplyd by \citet{Ricci:2008a}. This object is a very faint proplyd associated with a very diffuse shell. The northern bow wing is much more extended than the southern wing. The fact that the shell is so large and diffuse, may be an indication that it is not related to the proplyd, although the fact that a small cavity is seen  around the proplyd suggests that some degree of physical interaction is indeed occurring.

\subsection{North group}
\label{sec:n-group}

\textit{142-301.} The central source was first catalogued as cusp with tail \citep{ODell:1996a} and later as proplyd \citep{Bally:2000a, Ricci:2008a}. This large proplyd has one of the longest tails (\(4''\)) of any proplyd and does not have a hemispherical head. Instead, the ionization front appears to trace the disk surface, which is inclined with respect to the tail by about $55^{\circ}$. The proplyd tail points away from \(\mathrm{\theta^1\,Ori~A}\) instead of \thC~and exhibits some bends and wiggles \citep{Bally:2000a}. This proplyd is surrounded by a very faint emission arc.  Rather than showing continuous curvature, the emission arc appears to comprise two straight edges, which meet at a point south-east of the proplyd, with the shell being thicker on the southern side.

\textit{154-225.} The central source was previously catalogued as elongated body with diffuse boundary \citep{ODell:1996a}. Later, \citet{Ricci:2008a} reported it as proplyd in a binary sytem (the main body). The central source is sorrounded by a very faint and lumpy emission arc.

\textit{154-240.} The central source is a large bright proplyd \cite{Bally:2000a, Ricci:2008a} with a \(3''\) long tail and the inclined protoplanetary disk is seen in silhouette \citep{Bally:2000a}. There is evidence of an emission arc associated with this proplyd, but only the inner edge can be clearly traced. The outer edge of the shell merges with a thick knotty structure whose origin is unclear.

\textit{159-221.} This central source was first classified as a star by \citet{ODell:1996a}. This same source was reported as a dark disk seen only in silhouette by \citet{Ricci:2008a}. But a faint emission rim can be seen surrounding the disk in the \ha{} image, suggesting that it is an externally ionized proplyd. We identified an emission arc associated with the central star. The outer edge of the shell is very diffuse, which  makes it difficult to trace of outer rim. The axis of the bowshock is significantly deviates from the radial direction.

\textit{163-222.} The central proplyd was first catalogued by \citet{ODell:1996a}. There is a compact \(0.''15\) diameter disk seen nearly in face-on embedded in this stubby proplyd \citep{Bally:2000a, Ricci:2008a}.  This source was also reported as a binary sytem \citep{Ricci:2008a}. An a very faint and small emission arc wraps the proplyd. The outer and inner edges of the shell are well-defined on the eastern side, nevertheless the western side is superimposed on an unrelated a brighter larger scale emission filament, making impossible to trace the arc boundaries on this side.

\textit{170-249.} This central source is a previously catalogued proplyd \citep{ODell:1996a}, reported as a binary system by \citet{Ricci:2008a}. This bright and large proplyd exhibits a long tail and an inclined disk seen in silhouette \citep{Bally:2000a}. Several filamentary emission features with arc shape crosses the object, nonetheless it is possible to identify a very faint emission arc that seems to be associated to the proplyd.

\textit{178-258.}  This large and weak proplyd was catalogued by \citet{Ricci:2008a}, which is surrounded by a well-defined but faint emission arc.

\subsection{Northwest group}
\label{sec:nw-group}

\textit{4578-251.} This is a very bright T Tauri star associated with an emission arc that presents a double-shell morphology. This object has an asymmetric bow shock and the shell is more extended toward the south than to the north. The emission of the outer shell is fainter than the inner shell and is unclear whether the region marked with points is part of the outer shell.

\textit{049-143.} Figure shows a proplyd \citep{Ricci:2008a} with a very diffuse arc of emission. Its shell is thick, the wings of the bow shock are very open, and circular shaped. The inner edge is more diffuse than the outer edge and the bow shock is asymmetric. The  proplyd has a short tail, probably indicating that it is highly inclined and there is extinction in the center. 
 
\textit{051-024.} We identified a previously uncataloged proplyd with its bow shock located in front of the upper end of the North Bright
Bar. The emission shell is thin, but a second larger, and more diffuse, emission shell is seen in front of the bow shock that we have
marked. It is unclear whether this outer shell is related to the object or not. If it is, then the object would be similar to the nearby 072-134. 

\textit{072-134.} This object is located to the south east from 051-134. It was first cataloged by \citet{ODell:1996a} designated as 072-135. Later, \citet{Ricci:2008a} listed a disk seen nearly edge-on, also named 072-135. In this work a shell with a complex morphology is identify. The inner edge is a narrow bright arc, while the shell is thick and faint, which is only visible on the N~side.        

\textit{074-229.} This central source is located at south east of 073-227. It is a T Tauri star, that appears to be the smaller twin of the nearby 073-227, associated with a small and faint emission bow shock. The central star is not obviously a proplyd, but this may be because it is too small to be resolved. The projected separation from 072-134 is about \(8''.0\), which is not significantly smaller than the expected mean projected separation between nearest neighbors given the stellar density at this distance from the Trapezium \citep{Reipurth:2007a}, 
% http://adsabs.harvard.edu/abs/2007AJ....134.2272R
so the evidence that they form a physical binary pair is weak.

\textit{101-233.} The central source was first cataloged as proplyd by \citet{ODell:1996a}, designated as 102-233, and later by \citet{Ricci:2008a} with the same name. This proplyd is associated with a clumpy, thin and low ionization bow shock. Several additional broad filamentary emission features can be seen in front of the arc, but it is unclear if these are associated with the object or are a chance superposition, that also seem to be aimed toward Trapezium. 

\textit{102-157.} We identified a previously uncataloged proplyd associated with a very faint emission arc. The proplyd tail is very short, indicating that it is highly inclined. 102-157 has a open bow. The southwest wing of the bowshock is crossed by an apparently unrelated east-west oriented filament, which makes it difficult to trace the emission arc on this side.


\textit{106-245 and 109-246.} The bright proplyd 109-246
\citep{Ricci:2008a} was previously cataloged as 109-247 by
\citet{Bally:2000a}, who noted its central dark disk and possible
microjet.  The proplyd is situated just outside the HH~202 bow shock
and is superimposed on a complex background of fainter HH bow shocks
that point towards the north west.  However, a
south-east facing bow shock is also clearly visible just in front of
the proplyd.  We identify this as a stationary arc due to its
symmetric placement with respect to the proplyd axis and clear
morphological differences from the HH bow shocks.

A similar, albeit much smaller arc can be seen in front of the stellar
source 106-245.  Given the complex nebular background, the evidence
for an arc is not overwhelming, but is strengthened by the
morphological similarity with the nearby 109-246.  As with 074-249, the
source is not obviously a proplyd, but again this may be simply due to
its small size.  


\textit{124-131.} This object is a circumbinary proplyd, that was
first cataloged by \citet{ODell:1996a}, but they named it 124-132 and
classified as irregular. Later, this source was reported by
\citet{Smith:2005a} showing a microjet emerging perpendicular to the
major axis of the disk. \citet{Ricci:2008a} cataloged the source as a
binary system and \citet{Robberto:2008a} interpreted the central dark
silhouette as a circumbinary disk.  We report the discovery of a
broad, faint partial shell in front of the proplyd, which we identify
as a stationary arc.  The arc is barely visible in \ha{}, but is much
clearer in green and blue wideband filters, suggesting that we are
seeing dust-scattered continuum.  In this respect it is similar to the
arc in front of the close-in proplyd 163-317 (LV~3).   


\textit{132-053.}  Similar area to 131-046, and also may be affected
by extinction filaments.  The shell looks more convincing in this
object however, particularly the innner edge.


\textit{206-043.} This object appears to show an extremely faint
emission arc.  However, it is seen against a strongly variable nebula
background and a filamentary foreground extinction feature is seen
nearby (\(\approx 3''\) to west).  It is therefore possible that
the apparent arc is merely an ilusion caused by an extension of these
extinction filaments.  Although it is not cataloged as a proplyd, the
central source shows indications of a faint conical tail, pointing
almost due North, with a length \(\approx 2''\). 

\subsection{Southwest group}
\label{sec:sw-group}

\textit{4582-635.} This is a previously cataloged proplyd
\citep{Ricci:2008a}, which is surrounded by a very faint emission arc
that has not been reported previously in the literature. There are
hints of additional emission knots in front of the arc, but the S/N is
very low and it is unclear if they are related with the object.

\textit{4596-400.} \citet{Bally:2000a} first reported this object as a
``wind collision front'' with the designation 000-400. The central
source was identified as a proplyd by \citet{Ricci:2008a}, who gave it
the designation 4596-400 based on more accurate astrometry, which we
adopt here. The outer edge of the wings of the emission arc can be
traced to much farther distances than is typical, at least 5 times the
axial radius of curvature. The northern wing appears knotty at large
distances, whereas the southern wing is smooth. However, the knots
seen superimposed on the northern wing may be due to an unrelated,
larger scale filament that crosses both this object and 012-407, which
is most prominent in blue/green continuum filters.

\textit{042-628.} This source appears in the proplyd catalog of
\citet{Ricci:2008a} with the designation 038-627, supposedly based on
appearance in previous catalogs \citep{ODell:1996a}. However, we haver
been unable to locate the source in any other catalogs and the source
coordinates clearly indicate a designation of 042-628, which we adopt
here.  A bright compact knot is seen just inside the emission arc,
\(\approx 2''\) to the north of the proplyd, which may represent a
jet.  A fainter knot is also seen closer to the proplyd, but at a
slightly larger position angle.

\textit{LL 1 (056-519).} The T Tauri star LL Orionis is the prototype
of the LL objects. Its emission arc was discovered by
\citet{Gull:1979a} and is now denoted LL 1. This is a parabolic or
hyperbolic bowshock that wraps around the source star
\citep{Bally:2006a}. This emission arc was also reported by
\citet{Bally:2000a} and as a LL Orionis-type wind-wind collision
fronts LL 1 by \citet{Bally:2001a}. The emission of the bowshock wings
is blueshifted with repect to the backgraond nebular emission, the
emission from the head of the bowshock is at a similar velocity to the
nebula and shows no detectable proper motion, consistent with it being
a stationary structure \citep{Henney:2013a}. LL 1 is associated with a
hypersonic jet Herbig-Haro, HH 888, that arises in the T Tauri star.


\subsection{West group}
\label{sec:w-group}

\textit{4285-458.} This is an emission arc that wraps a bright T~Tauri
star. Although the outer boundary of the shell is well-defined, it is
impossible to trace the shell's inner boundary due to confusion with
the PSF wings of the central star. This is the most distant emission
arc from the Trapezium in this catalog and It is also much smaller
than the other arcs in the west group.

\textit{LL 3 (4408-639).} This previously reported LL~Ori-type object
\citep{Bally:2001a} exhibits a double-shell morphology to the emission
arc: an inner shell that is brighter and narrower, and an outer shell
that is fainter and broader, but still with a well-defined outer edge.
In addition, the central star shows a faint emission structure, which
protrudes to the WSW, and which may represent a proplyd tail, although
the object is not included in any proplyd catalogs, such as
\citet{Ricci:2008a}.

\textit{LL 2 (4409-242).} This large LL~Ori-type object
\citep{Bally:2001a}, is associated with the T~Tauri star IX~Ori. It
also has a bipolar jet, HH~505, which is oriented nearly perpendicular
to the bowshock axis. Apart from LL~1, this is the only LL object
whose kinematics have been studied via spectroscopic mapping
\citep{Henney:2013a}. Unlike LL~1, the structure and kinematics of the
LL~2 arc are very asymmetrical.  Additionally, proper motion studies
\citep{Henney:2013a} show that only the head and northern wing of the
arc are stationary structures: the southern portion of the arc has
high proper motion and seems to be driven by the blueshifted jet.

\textit{LL 4 (4427-838).} This is a previously reported LL~Ori-type
object \citep{Bally:2001a}. The central source was reported as a
proplyd and a binary system by \citet{Bally:2006a}. At first glance,
this object appears to be a single arc with a very large radius of
curvature. However wide closer inspection suggests that the ``wings''
of the arc are separate structures from the more curved nose
region. We suggest that they may not be part of the true LL arc, but
are instead driven by the bipolar jet \citep{Bally:2006a} in a similar
way to the southern wing of LL~2.

\textit{4468-605.}  This arc was first reported by \citet{Bally:2006a}
who also noted the proplyd nature of the central source and the
presence of a microjet (HH 886) oriented approximately parallel to the
symmetry axis of the proplyd and arc.  Although \citet{Bally:2006a}
report the jet as one-sided, a bright \ha{} filament seen at the end
of the proplyd tail may represent the counterjet.  The north side of
the emission arc shows an apparent flared morphology, but this may be
illusory, due to an unrelated larger scale emission filament that is
projected against that side of the bowshock.

\subsection{South group}
\label{sec:s-group}

\textit{066-3251.} This object was classified as a proplyd by \citet{Ricci:2008a} and, located at a distance \(\sim 10'\) south of the ONC core, it is one of the farthest known proplyds from the Trapezium. We have identified a faint arc associated with this proplyd, which is seen most clearly in the F555W broad band filter but is also visible in the \ha{} filter. This object is projected onto the tip of a large-scale wishing bone shaped filament, which seems to represent a local ionization front. We believe that the ``outflow'', which \citet{Ricci:2008a} identify to the south of this object is merely a misidentification of part of this filament.  

\textit{116-3101.} Figure shows a proplyd associated with a small but sharply defined emission arc.  The central star is also named V488 Ori \citep{Bally:2006a}. The wings of the bow shock are very closed, the outer edge is well defined and has a circular shape since we can fit a perfect circle with the points marked.  

\textit{119-3155.} This is a binary system  associated with an emission arc. Based on the position of the outer of curvature of the arc, we assign the fainter, more northern binary component as the corresponding stellar source. It has a faint arc to the north and another to the east (Fig), but  the east arc is probably unrelated to the object, and may instead be associated with the HH 880 bowshock which passes \(15''\) to the south. This emission arc has not been previously reported in the literature.     

\textit{136-3057.} This is a T Tauri star associated with an emission arc. It is located to the north of 119-3155 and is one of the bigger objects of this group. 136-3057 has a very diffuse shell. This has not been previously reported in the literature.

\textit{138-3024.} Figure shows a T Tauri star associated with a thin shell. The arc is seen most strongly blue and green broad band filters, but it is also seen at very low contrast in the \ha{} filter. This object has not been previously reported in the literature.

\textit{203-3039.} This object has a microjet (HH 561) which was discovered in the Fabry-Pérot study of the southern Orion Nebula by \citet{Bally:2001a}. Later, this HH object was described in more detail by \citet{Bally:2006a}, the microjet emerges toward the east from the variable star MY Ori, terminating in a faint bowshock at a distance of \(16''\). A fainter counterjet and bowshock are located west of the star. In addition to these previously reported features, we identify an LL-type emission arc associated with MY Ori. The faint bow is very open.

\textit{261-3018.} This is a relatively large and diffuse LL-type arc previously unreported, seen superimposed on a very complicated region of overlapping flows. We tentatively identify the star 261-3018 as the source, although the star (264-3016) is another possible candidate. In the \ha{} the 261-3018 star (which is the same source as 262-3018 reported by \citealp{Bally:2006a})  shows a linear protrusion of length \(\simeq1''\) towards the southwest. This is unlikely to be a proplyd tail, since it is not aligned with the radial direction from the Trapezium, and instead may be a microjet. Indeed, a series of faint knots and bowshocks extend for \(30''\) in the same direction. The LL-type arc is very flat and shows a bright rim at its inner edge. It extends further to the east than to the west. The object is seen in projection superimposed on the HH 502 flow but shows no evidence of physical interaction with this flow. 

\textit{362-3137.} This is a previously uncatalogued proplyd associated with an emission arc. However, \citet{Da-Rio:2009a} catalogued the central source as a star. The source is a double star. It is located at north of LL 7. This object seems to show a double-shell morphology. There is a filament to south probably unrelated to the object. This emission arc has not been previously reported in the literature. 

\subsection{Ambiguous objects}
\label{sec:problematic-objects}

\textit{065-502.} This was classified as a non-proplyd stellar source
by \citet{ODell:1996a}. However, the star has a small protrusion
pointing away from Trapezium, which may indicate a proplyd tail. This
proplyd tail is very short, suggesting that the object it is highly
inclined.  A very faint and diffuse arc can be discerned in front of
the object, but it is superimposed on an irregular background, which
makes it difficult to delineate its form.

\textit{066-652.}  This is in the distant south of the nebula and is
so faint that it is impossible to tell whether the nebulosity
associated with the star is a stationary arc or instead a
bright-rimmed globule. 

\textit{083-435.} The central source shows a very faint tail
suggesting that this source is a previously uncatalogued proplyd.
There is slight evidence for a broad diffuse emission arc in front of
the proplyd, but the source is near to the ``SW Cloud'' foreground
extinction feature \citep{ODell:2000a}, and the presence of faint
extinction filaments complicates the identification.


\textit{117-421.} This small proplyd \citep{Ricci:2008a} appears to
excavate a small cavity around it, but no shell outer edge is
evident.  Additionally, as for 083-435, the proximity to foreground
extinction filaments may be a confusing factor. 

\textit{121-434.}  Another small proplyd in the same area, with the
appearance of a clump emission arc in front of it.  However, once
again, the highly variable extinction makes the identification
uncertain. 

\textit{131-046.}  There is the appearance of a very broad and flat
shell in front of this proplyd.  However, it is in a region that is
criss-crossed by multiple extinction filaments and the apparent shell
may be an illusion. 


\textit{173-236.} This large proplyd is has the weak evidence for an
emission arc in front of it, but it has a very small radius of
curvature compared with the other arcs.  It is located near to the
``Dark Bay'' \citep{ODell:2000a} and is crossed by multiple extinction
filaments, which make the identification uncertain.

\textit{212-400.}  A small proplyd with weak evidence for an emission
arc in front of it.  As with 173-236, it is near to the ``Dark Bay''
and the apparent arc may be an illusion caused by inhomogeneous
foreground extinction.



